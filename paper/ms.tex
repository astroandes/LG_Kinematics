\documentclass{emulateapj}
\submitted{{\it Submitted for publication in ApJ Letters}}
\usepackage{multirow,color,wrapfig,ulem}
\usepackage {graphicx}
\bibliographystyle{apj}
\usepackage{graphics}
\usepackage[dvips]{epsfig}

\newcommand{\kms}{\,km~s$^{-1}$}
\def\squig{\sim\!\!}
\newcommand{\LCDM}{$\Lambda$CDM~}
\newcommand{\beq}{\begin{eqnarray}}  
\newcommand{\eeq}{\end{eqnarray}}  
\newcommand{\zz}{$z\sim 3$} 
\newcommand{\avg}[1]{\langle{#1}\rangle}  
\newcommand{\ly}{{\ifmmode{{\rm Ly}\alpha}\else{Ly$\alpha$}\fi}}
\newcommand{\hMpc}{{\ifmmode{h^{-1}{\rm Mpc}}\else{$h^{-1}$Mpc }\fi}}  
\newcommand{\hGpc}{{\ifmmode{h^{-1}{\rm Gpc}}\else{$h^{-1}$Gpc }\fi}}  
\newcommand{\hmpc}{{\ifmmode{h^{-1}{\rm Mpc}}\else{$h^{-1}$Mpc }\fi}}  
\newcommand{\hkpc}{{\ifmmode{h^{-1}{\rm kpc}}\else{$h^{-1}$kpc }\fi}}  
\newcommand{\hMsun}{{\ifmmode{h^{-1}{\rm {M_{\odot}}}}\else{$h^{-1}{\rm{M_{\odot}}}$}\fi}}  
\newcommand{\hmsun}{{\ifmmode{h^{-1}{\rm {M_{\odot}}}}\else{$h^{-1}{\rm{M_{\odot}}}$}\fi}}  
\newcommand{\Msun}{{\ifmmode{{\rm {M_{\odot}}}}\else{${\rm{M_{\odot}}}$}\fi}}  
\newcommand{\msun}{{\ifmmode{{\rm {M_{\odot}}}}\else{${\rm{M_{\odot}}}$}\fi}}  
\newcommand{\lya}{{Lyman$\alpha$~}}
\newcommand{\clara}{{\texttt{CLARA}}~}
\newcommand{\rand}{{\ifmmode{{\mathcal{R}}}\else{${\mathcal{R}}$ }\fi}}  
\newcommand{\Lsun}{\mbox{\,$L_{\odot}$}}
\newcommand{\like}{\mathscr{L}}
\newcommand{\bftheta}{\mathbf{\Theta}}
\newcommand{\degree}{\ensuremath{^\circ}}
\def\spose#1{\hbox to 0pt{#1\hss}}
\def\simlt{\mathrel{\spose{\lower 3pt\hbox{$\mathchar"218$}}
     \raise 2.0pt\hbox{$\mathchar"13C$}}}
\def\simgt{\mathrel{\spose{\lower 3pt\hbox{$\mathchar"218$}}
     \raise 2.0pt\hbox{$\mathchar"13E$}}}
\font\smcap=cmcsc10
\def\caii{Ca\,{\smcap ii}}


\shorttitle{The not so common kinematics of the Local Group}
\shortauthors{Forero-Romero et al.}

\begin{document}
\title{The not so common kinematics of the Local Group}
\author{
J. E.\ Forero-Romero\altaffilmark{1}, 
Y. Hoffman\altaffilmark{2}, 
S.  Bustamante\altaffilmark{3}, 
S. Gottl\"ober\altaffilmark{4}, 
G. Yepes\altaffilmark{5}
}

\altaffiltext{1}{Departamento de F\'{i}sica, Universidad de los Andes, Cra. 1 No. 18A-10, Edificio Ip, Bogot\'a, Colombia, \email{je.forero@uniandes.edu.co}}
\altaffiltext{2}{Racah Institute of Physics, The Hebrew University of Jerusalem, 91904 Jerusalem, Israel}
\altaffiltext{3}{Instituto de F\'{\i}sica - FCEN, Universidad de Antioquia, Calle 67 No. 53-108, Medell\'{\i}n, Colombia}
\altaffiltext{4}{Leibniz-Institut f\"ur Astrophysik, Potsdam, An der Sternwarte 16, 14482 Potsdam, Germany}
\altaffiltext{5}{Grupo de Astrof\'{\i}sica, Departamento de F\'{\i}sica Te\'orica, Universidad Aut\'onoma de Madrid,
Cantoblanco E-280049, Spain}

\date{\today}

\begin{abstract}
   
Recent observations constrained the tangential velocity of M31 with respect to the Milky Way (MW) to be $v_{\rm M31,tan}<34.4$ \kms and the radial velocity to be in the range $v_{\rm M31,rad}=-109\pm 4.4$\kms \citep{vanderMarel12}. In this study we use a large volume high resolution N-body cosmological simulation to statistically study this kinematics in the context of the $\Lambda$CDM cosmology. We define the Local Group (LG) to be composed by two dominant separate halos hosting the MW and the M31 galaxy fulfilling certain isolation criteria. We find that the most probable values for the tangential and radial velocities in these pairs are $v_{\rm rad, \Lambda CDM}=-65\pm 5$\kms and $v_{\rm tan, \Lambda CDM}=62\pm 5$\kms. Within a similar absolute uncertainty defined by observations the pairs centered around these values are $\sim5$ times more abundant than the pairs in the observational interval. Furthermore, we find that only $\sim8\%$ of the pairs show a ratio between its tangential and radial velocity consistent with observations. {\bf This tension increases if we make a narrower selection in the pair sample to match the observational constraints on pair separation and total mass. Expressing these results in terms of the angular momentum and mechanical energy per unit mass also reflects this disagreement albeit in a less strong fashion due to the larger uncertainties inferred for these physical quantities. We also include in our analysis results from constrained simulations and cosmic web charachterization techniques. The results of this study imply that the formation and evolution of the LG differ from the average $\Lambda$CDM halo pair.}

\end{abstract}

\begin{keywords}
{galaxies: kinematics and dynamics, Local Group, methods:numerical}
\end{keywords}

\section{Introduction}

The Milky Way (MW) and Andromeda galaxy (M31) are the dominant galaxies in the Local Group (LG). Astronomical observations of their mass distribution impose constraints on the standard cosmological model. The satellite overabundance problem \citep{Klypin99,Moore99}, tidal disruption features \citep{pandas09} and the disk dominated morphology \citep{Kazantzidis2008} are examples on of how LG studies are linked to the cosmological context. Detailed studies on the Magellanic Clouds dynamics and their possible link to M31 add to the interest of understanding the details of the LG kinematics and dynamics \citep{Besla2007,Tollerud2011,Knebe2011,Fouquet2012,Teyssier2012}. However, a general concern in the use of the LG as a tool for near-field cosmology \citep{Freeman2002,Peebles2010} is how typical is the LG regarding the properties of interest \citep{Liu2011,ForeroRomero2011,Purcell2012}. 

A new valuable piece of information in this issue is the recent observational determination of the proper-motion measurements of M31, which until recently had been out of reach \citep{vanderMarel12}. This measurement opens up the possibility to study in detail the dynamics of the LG in a cosmological context. The reported measurements set an upper bound for the tangential velocity of M31 with respect to the MW of $v_{\rm tan,M31}\leq 32$ km s$^{-1}$. Together with the values of the relative radial velocity of $v_{\rm rad,M31}=-109\pm 4.4$ \kms observations show that the relative motion of the MW and Andromeda is consistent with a head-on collision. With this information it is possible to ask how common is this kinematic configuration in a $\Lambda$CDM Universe.

This Letter presents such study. We use a large volume, high resolution dark matter only N-body simulation in the concordance $\Lambda$CDM cosmology to find a set of halo pairs with similar isolation criteria as inferred in the LG. We quantify these results in terms of the number of pairs with given radial and tangential velocities in the galactocentric rest frame. We also find the pairs that are consistent with a head on collision in terms of the ratio of the radial to tangential velocity $f_{\rm t}\equiv v_{\rm tan}/v_{\rm rad}<0.3$ and present these results in terms of the reduced angular momentum and mechanical energy.

In addition we make use of three N-body simulations with constrained initial conditions which are constructed to reproduce the observed large scale structure of the Local Universe on scales of a few tens of Mpc. The special feature of these simulations is that each volume features a pair of halos with the right isolation and relative positioning characteristics to be considered Local Group halos.  Finally, we provide an interpretation of these results in terms of the large scale structure environment. We find what places in the cosmic web favor the presence of dark matter halos with kinematics similar to the LG.

This Letter is structured as follows. In the next section we present the N-body simulations, the criteria we use to select LG-like halo pairs and the method we use to characterize the cosmic web. In Section \ref{sec:results} we present the results for the dynamics in these pairs in terms of the tangential/radial velocities and the orbital angular momentum/mechanical energy. In the same section we investigate what kind of large scale environment can foster the production of pairs with dynamics similar to those observed in the LG. Finally, in the last section we comment and conclude about the implications of these results in the context of the $\Lambda$CDM model.

\section{Simulation and Environment}
\label{sec:methods}
\subsection{The Bolshoi and Constrained Simulations}





\label{subsec:lg-sample}
\begin{figure*}
\begin{center}
\includegraphics[keepaspectratio=true,width=0.45\textwidth]{./figures/separation.pdf}
\includegraphics[keepaspectratio=true,width=0.45\textwidth]{./figures/total_mass.pdf}
\caption{\label{fig:distros} Distributions of separations and total halo mass (Milky Way and M31) in the LG-like pairs selected from the Bolshoi Simulation.}
\end{center}
\end{figure*}


The Bolshoi simulation follows the non-linear evolution of the dark matter density field using N-body techniques. The simulation has a cubic volume of $250$\hMpc comoving on a side, sampled with $2048^{3}$ particles. The cosmological parameters used in the simulation are $\Omega_{m}=0.27, \Omega_{\Lambda}=0.73, \sigma_{8}=0.82, h=0.70$ and $n=0.95$, corresponding to the matter density, vacuum energy density, the normalization of the power spectrum, the reduced Hubble constant and the index of the slope in the initial power spectrum. This set of parameter is compatible with the analysis  of the seventh year of data from the Wilkinson Microwave Anisotropy Probe (WMAP) \citep{Jarosik2011}. A detailed description of this simulation can be found in \citep{Bolshoi}.

With these parameters the mass per particle is $m_{p}=1.4\times 10^{8}$\hMsun. In this paper we use the halos obtained through a FOF algorithm with a linking length $0.17$ times the average interparticle distance. We have obtained the data through the public available Multidark database \footnote{{\tt http://www.multidark.org/MultiDark/}} \citep{2011arXiv1109.0003R}. The database allows us to obtain the comoving positions, peculiar velocities and masses for all the halos in the simulation volume at $z=0$. The Hubble flow is taken into account to convert the relative velocities into physical velocities and allow for a comparison with observations.

The constrained simulations we use in this Letter are part of the Constrained Local UniversE Simulations (CLUES) project whose main objective is to reproduce the large scale structure in the Local Universe as accurately as observations permit. The algorithm and observational constraints to construct the initial conditions are described in \citep{clues2010}.  We use three dark matter only simulations, each has a cubic volume of $64$\hMpc on a side, with the density field sampled with $1024^3$ particles. The cosmological density parameter is $\Omega_m=0.28$, the cosmological constant $\Omega_{\Lambda}=0.72$, the dimensionless Hubble parameter $h=0.73$, the spectral index of the primordial density perturbations $n=0.96$ and the power spectrum normalization $\sigma_{8}=0.817$, also consistent with WMAP 7th year data. The halo catalogs are constructed with the same FOF code used on the Bolshoi Simulation.

\subsection{The LG-sample}
Based on the FOF catalogs in the Bolshoi simulation we construct a halo pairs sample with masses and isolation properties consistent with the inferred properties for the MW and M31. The characteristics we impose to define a LG-like halo pair are the following:

\begin{enumerate}
\item Each halo has a mass in the range $5\times10^{11}\hMsun <M_{h}<5\times 10^{12}\hMsun$.
\item With respect to each halo, there cannot be any other halo within the mass range $5\times10^{11}\hMsun <M_{h}<5\times 10^{12}\hMsun$ closer than its partner. It means that there cannot be ambiguity on the identity of the pair members.
\item The relative radial velocity between the two halos is negative \citep{vanderMarel12}.
\item The distance between the center of mass of the halos must be less than $0.7$\hMpc \citep{ribas05,vanderMarel08}.
\item There cannot be halos more massive than $5\times 10^{12}$\hMsun within a radius of $2$\hMpc with respect to every object centre \citep{Karachentsev04,Anton09}.
\item There cannot be halos more massive than $5\times 10^{13}$\hMsun with  a radius of $3$\hMpc with respect to every object centre \citep{Karachentsev04}.
\end{enumerate}

This sample in the Bolshoi simulation has $1256$ pairs. This is $\sim 30\%$ smaller than the the number of pairs we found in \cite{ForeroRomero2011}, which did not include the condition 5) in the construction of the sample. Additionally, there is a sample of three (3) pairs constructed from the three constrained realizations. These pairs fulfill all the above mentioned conditions and additionally are located in a place with the right distances with respect to the Virgo and Coma clusters in the simulation. Figure \ref{fig:distros} show the distribution of separations and total pair mass computed from the $1256$ pairs in the Bolshoi simulation. 

{\bf Additional constraints will be set on this population in order to match the observational estimates for separation and total mass. The full observational characteristics that we take in this Letter for the MW-M31 pair are listed in Table \ref{table:1}.}



\begin{table}
\caption{Summary of the kinematic properties for M31 in the galactocentric frame as reported by \citep{vanderMarel12}. Values in parenthesis correspond to vector components}
\begin{center}
\begin{tabular}{ccc}\hline\hline
${\bf r}_{\rm M31}$ & (kpc) &$(-378.9, 612.7, -283.1)$\\
$\sigma_{{\bf r}, {\rm M31}}$ & (kpc) &$(-18.9, 30.6, 14.5)$\\
${\bf v}_{\rm M31}$ & (\kms) & $(66.1, -76.3, 45.1)$\\
$\sigma_{{\bf v},{\rm M31}}$ & (\kms) &$(26.7, 19.0, 26.5)$\\
$v_{\rm M31,rad}$ &(\kms) & $-109.3\pm 4.4$\\
$v_{\rm M31,tan}$ &(\kms) & $<34.4$\\
$r_{\rm M31}$ &(kpc) & $770\pm 40$\\
$M_{\rm 200,MW} + M_{\rm 200, M31}$ & ($10^{12}\Msun$) & $3.14\pm 0.58$\\\hline
\end{tabular}\\
\vspace{1mm}
Note: the observational uncertainties in the position vector correspond to a $5\%$ in each component consistent with the uncertainties in the distance \citep[see references in][]{vanderMarel08}
\end{center}
\label{table:1}
\end{table}


\subsection{The kinematic environment}
We implement a large scale environment classification to quantify the dynamical state of the dark matter field. A detailed description of the algorithm can be found in \cite{Hoffman12}. Here we briefly describe its main physical assumptions.

The classification is based on computing the following velocity shear over a grid:

\begin{equation}
\Sigma_{ij}=-\frac{1}{2H_{0}}\left(\frac{\partial v_{i}}{\partial r_{j}}+\frac{\partial v_{j}}{\partial r_{i}}\right),
\end{equation}

where $i,j=x,y,z$, $H_{0}$ is the Hubble constant and $v_{i}, r_{i}$ correspond to the peculiar velocities and comoving positions, respectively. The matrix  $\Sigma_{ij}$ is symmetric and real, allowing for its diagonalization with real eigenvalues $\lambda_{1}>\lambda_{2}>\lambda_{3}$. A positive value of $\lambda_{i}>0$ implies collapse in comoving coordinates along the direction defined the corresponding eigenvector $u_{i}$. This method allows for the classification of each point as a peak, filament, sheet and void, if there are three, two, one and zero eigenvalues larger than a given threshold $\lambda_{\rm th}$, respectively.


Environment classification is an intrinsic multi-scale problem. Morphological features can appear or disappear depending on the physical scale that defines the density and velocity field. In the case of the LG kinematics there is a physical scale defined by the typical separation between the MW and M31, which is on the order of $\sim 0.5$ \hMpc. For this reason, all the results for the environment that we present here are computed over a grid of $256^3$ cells with a spacing of $0.97$\hMpc, where a first CIC interpolation is performed and later a Gaussian smoothing with a variance scale of $0.97$\hMpc is applied. In this Letter we are interested in the possible range of $\lambda_{i}$ spanned by the halo pairs that display dynamics similar to those of the observed LG. 


\section{Results}
\label{sec:results}



\begin{figure*}
\begin{center}
\includegraphics[keepaspectratio=true,width=0.46\textwidth]{./figures/test_rt.pdf}
\includegraphics[keepaspectratio=true,width=0.46\textwidth]{./figures/test_narrow_rt.pdf}
\caption{Histograms of the radial and tangential velocities for LG-like halo pairs in the Bolshoi simulation. The left panel corresponds to the general LG-like pair simple, the right panel corresponds to the subsample with cuts on pair separation and total mass, matching the observational constraints. The squared panel shows as a shaded histogram the number of pairs on the radial and tangential velocities plane. The rectangular panels show the histograms when only one of the velocity components is used to bin the data. The half ellipsoid corresponds to the observed observational constraints for M31 in the Galactocentric rest frame: $v_{\rm rad}=109.2\pm 4.4$ km s$^{11}$ and $v_{\rm tan}< 34.3$ km s$^{-1}$ \citep{vanderMarel12}. The circles represent the positions of the pairs from the constrained simulations. The highest number density in the plot is obtained around $v_{\rm rad,\Lambda CDM} = 65\pm5$ \kms, $v_{\rm tan,\Lambda CDM} = 62\pm5$ \kms}
\label{fig:rt}
\end{center}

\end{figure*}


\subsection{Radial and tangential velocities}

Figure \ref{fig:rt} summarizes the central finding of this Letter. Most of the LG-like pairs (left panel) in the Bolshoi Simulation and the pairs in the constrained simulation have radial and tangential velocities notably different from the observational constraints.  The most probable velocities in the simulation are around $v_{\rm rad,\Lambda CDM} = -65\pm5$\kms and $v_{\rm tan,\Lambda CDM} = 62\pm5$ \kms, where the uncertainties in these values reflect the minimum grid size needed to obtain robust statistics from the 2D histogram. There are $45$ pairs in the region centered around those values within an interval width comparable with the uncertainties in the observational constraints ($\sigma_{\rm tan}=17$\kms and $\sigma_{\rm rad}=4$\kms.) In contrast, there are only $8$ pairs in the interval allowed by observations. This makes the LG-like pairs consistent with observations $\sim5$ times less likely to be found than a pair with velocities around $v_{\rm rad,\Lambda CDM}-v_{\rm tan,\Lambda CDM}$.

From Figure \ref{fig:rt} it is also clear that there is a significant number of pairs with a high tangential-to-radial velocity ratio. The peak in the pair number density is located around a region of $f_{\rm t}\equiv v_{\rm tan \Lambda CDM}/v_{\rm rad \Lambda CDM}\sim 1$, while the observations suggest $f_{\rm t}<0.32$. In the Bolshoi Simulation, out of $1256$ LG-like pairs we find that $108$ pairs ($\sim 8\%$ of the total sample) have values for $f_{\rm t}$ consistent with that constraint. {\bf The three pairs from the constrained realizations are also outside the region of the most probable values in a $\Lambda$CDM cosmology with tangential-to-radial ratios of $f_{\rm t}= 0.35, 0.45, 0.73$.}

The right panel in Figure \ref{fig:rt} shows the results of making narrower selection of the LG-like pairs to separations between $700-800$ kpc and a total mass in the two halos to be in the range $1-4 \times 10^{12}$\Msun. This change emphasizes the tension found from the full LG-like sample. The sample is smaller ($1256$ pairs are reduced to $180$) and the values corresponding to the highest density region in the radial-tangential velocity plane are  $v_{\rm rad,\Lambda CDM} = -36\pm 5$\kms and $v_{\rm tan,\Lambda CDM} = 50\pm5$\kms with $10$ pairs around that value and zero pairs consistent with observations. This shows that the atypical LG dynamics in $\Lambda$CDM and the low fraction of radially dominated velocities are robust results with respect to the separation and total mass conditions used to define the LG-like pairs. {\bf We note that this tighter selection also excludes 2 of the 3 pairs from the constrained simulations.}


\subsection{Reduced Angular Momentum and Energy}


To gain additional physical insight we consider the two halos as points of mass $m_{\rm M31}$ and $m_{\rm MW}$. In the center of mass their orbit has an angular momentum $L=\mu|{{\bf r}_{\rm M31}}\times{\bf v}_{\rm M31}|$ and mechanical energy $E=\frac{1}{2}\mu |{\bf v}_{\rm M31}|^2 - G\mu M/|{\bf r}_{\rm M31}|$, where the total mass is $M=m_{\rm M31}+m_{\rm MW}$, the reduced mass is $\mu=m_{\rm M31}m_{\rm MW}/M$ and $G$ is the gravitational constant. We express the results of the previous section in terms of the reduced angular momentum and energy $l_{orb}=L/\mu$ and $e_{\rm tot}=E/\mu$. 

{\bf This formulation has a clear theoretical advantage if one considers the angular momentum and the mechanical energy as quasi-conserved dynamical quantities. This means that, after some formation time, these quantities do not significantly evolve as it is the case for the radial and tangential velocities. But there is an observational disadvantage. These quantities have to be derived from observations, increasing the uncertainties in their determination.}

In Table \ref{table:1} we list the M31 position (${\bf r}_{\rm M31}$), M31 velocity (${\bf v}_{\rm M31}$) and total mass ($M$) that we use to estimate the uncertainties in the $e_{\rm tot}-l_{\rm orb}$ plane. To do this we generate $10^6$ pairs with different values for these three variables using a Gaussian distribution for each vector component with a variance equal to their uncertainties ${\sigma_{\bf r}}_{\rm M31}$ and ${\sigma_{\bf v}}_{\rm M31}$).. For each pair we calculate its $l_{\rm orb}$ and $e_{\rm tot}$ values.  

The results are summarized in Figure \ref{fig:EJ} following the same plotting conventions from Figure \ref{fig:rt}. The closed line is the iso-density contour in the 2D histogram enclosing $68\%$ of the Monte Carlo generated pairs consistent with observations. The poor observational constraints on the $e_{\rm tot}-l_{\rm orb}$ plane in comparison to the radial-tangential velocity plane comes from the fact that we are adding up large uncertainties in three different aspects. First, the tangential velocity has a $100\%$ uncertainty, making the angular momentum compatible with very low and high values. Second, the uncertainty on the square of the norm of the velocity has an uncertainty of $40\%$, having an impact on the kinetic energy. Third, the uncertainty in the total mass of the Local Group is close to $20\%$, which has an impact on the potential energy. 

In the right panel of Figure \ref{fig:EJ} we see that further selection of LG-like pairs to have similar separation and total mass makes them occupy a narrower region. This shows that there are preferred values for $e_{\rm tot}-l_{\rm orb}$. Under the assumption of quasi-conserved quantity it means that their formation and evolution conditions must be similar. In this case the agreement with observations is moderate. A reduction by a factor of $2$ in the uncertainty of the tangential velocity showing that the tangential velocity is below $17$\kms would clearly bring the observations in the region of low $l_{\rm orb}$, clearly below the average $\Lambda$CDM expectation.








\subsection{Environment}


\begin{figure*}
\begin{center}
\includegraphics[keepaspectratio=true,width=0.46\textwidth]{./figures/test_EJ.pdf}
\includegraphics[keepaspectratio=true,width=0.46\textwidth]{./figures/test_narrow_EJ.pdf}
\caption{Histograms of the orbital angular momentum ($l_{\rm orb}$) and mechanical energy ($e_{\rm tot}$) per unit of reduced mass calculated considering the halos as point masses. The panel distribution is the same as in Figure \ref{fig:rt}. The contour line shows the $1-\sigma$ uncertainty estimated through Monte Carlo techniques from the observational values summarized in Table 1. The poorer constraints on this plane are due to the added uncertainties on the tangential velocity ($100\%$), the square of the norm of the velocity ($40\%$) and the total mass of the two halos ($20\%$). }
\label{fig:EJ}
\end{center}
\end{figure*}

\begin{figure}
\begin{center}
\includegraphics[keepaspectratio=true,width=0.50\textwidth]{./figures/test_lambda.pdf}
\caption{Integrated probability distributions for the three eigenvalues in the V-web environment classification algorithms. The dark region marks the eigenvalue interval defined by the nine (9) pairs with similar kinematics as observed in the Local Group.}
\label{fig:lambda}
\end{center}
\end{figure}


We compute the V-web eigenvalue distribution for all halos more massive than $10^{11}$\hMsun in the Bolshoi Simulation. Figure \ref{fig:lambda} illustrates the results of this exercise. The solid line shows the integrated distribution for the three eigenvalues. The shadow regions are defined by the minimum and maximum of the eigenvalues in the nine (9) pairs in the LG-like sample that show similar kinematics as the observations. 


There is a preference for the largest eigenvalue to be positive  and have an upper bound $\lambda_{1}<1$, the second largest eigenvalue is bounded between $-0.1 < \lambda_{2} < 0.3$, while the smallest eigenvalue is always negative with a lower bound $\lambda_{3}>-0.5$. Under these conditions the peak and the void environment are completely excluded as environmental hosts of LG-like pairs for reasonable values of the threshold eigenvalue $\lambda_{\rm th}\approx 0.2-0.4$ \citep{Hoffman12,Libeskind12}. All pairs with kinematics similar to observations must reside in filaments or sheets. The result is similar if we consider instead the $108$ pairs with a weak tangential velocity component $f_{\rm t}<0.32$. 

However, the inverse condition is weak. Making a selection of pairs by constraining the eigenvalues $\lambda_{i}$ to the interval mentioned above, does not bias significantly the values for the radial and tangential velocities. This suggests that residing in filaments/sheets is a condition common to isolated pairs, regardless of their kinematics. A detailed study on the influence of environment in LG-like pairs will be presented in a forthcoming publication (Bustamante et al. {\it in prep}).



\section{Conclusions}
We have presented a comparison between the observed kinematics for the M31 in the galactocentric restframe and the expectations for a large N-body cosmological simulation in the $\Lambda$CDM cosmology. In the simulation we select a sample of halo pairs in the mass range $5\times 10^{11}<M_{h}/\hMsun<5\times 10^{12}$ that closely match the isolation conditions of the Local Group from other massive structures. While the observations show that M31 moves towards us with a highly radial velocity, the simulation shows that the most common configuration at $z=0$ has values $v_{\rm rad, \Lambda CDM}=-65\pm5$\kms and $v_{\rm tan, \Lambda CDM}=62\pm5$ \kms. 


Using the same absolute values for the uncertainty in the observed velocity components, we find that halos within the preferred $\Lambda$CDM values are five times more common than pairs compatible with the observational constraint.  The qualitative nature of these results is still valid after a narrower selection on separation and total pair mass. Additionally, pairs with a fraction of tangential to radial velocity $f_{\rm t}<0.32$ (similar to observations) represent $8\%$ of the total sample of LG-like pairs. Making an tighter selection to match the observational constraints on the separation and total mass results in zero pairs compatible with observations.


Approximating the LG as two point masses we express the above mentioned results in terms of the orbital angular momentum $l_{\rm orb}$ and the mechanical energy $e_{\rm tot}$ per unit of reduced mass. We find that the uncertainties in the tangential velocity, the square of the norm of the velocity and the total mass in the LG produce poorer constraints on the number of simulated pairs that are consistent with the observations. Nevertheless, in the case of the LG-pair sample that also fulfills the separation and total mass criteria there is a slight tension between simulation and observation. A reduction by a factor of $2$ in the observational uncertainty on the radial velocity would clarify this issue.

{\bf In the three pairs from constrained simulations we find kinematics dominated by radial velocities. However their velocity components differ from the observational constraints and their mechanical energy and orbital angular momentum are in modest concordance with observations. This conforms to the fact that the observational values do not seem to be common in a $\Lambda$CDM Universe. Additionally we have found that pairs with velocities dominated by the radial component (by the same amount allowed by observations) are preferably to be found in filaments and sheets of dark matter. }

{\bf In this Letter we have shown that LG-like pairs in $\Lambda$CDM show prefered values for their relative velocities, angular momentum and mechanical energy. These values are in tension with the observational results. Pairs with highly radial orbits are not common in $\Lambda$CDM. This hints that the formation and evolution of the Local Group had special characteristics with respect to the average expectations in $\Lambda$CDM. Understanding the origin and implications of this result deserve further study.}



\label{sec:conclusions}
\section*{Acknowledgments}  
J.E.F-R acknowledges financial support from the Vicerrector\'{\i}a de Investigaciones at Universidad de los Andes through its {\it Fondo de Apoyo a Profesores Asistentes} and the Peter and Patricia Gruber Foundation through its fellowship administered by the International Astronomical Union.


\bibliographystyle{apj}
%\bibliography{references} 
\begin{thebibliography}{25}
\expandafter\ifx\csname natexlab\endcsname\relax\def\natexlab#1{#1}\fi

\bibitem[{{Besla} {et~al.}(2007){Besla}, {Kallivayalil}, {Hernquist},
  {Robertson}, {Cox}, {van der Marel}, \& {Alcock}}]{Besla2007}
{Besla}, G., {Kallivayalil}, N., {Hernquist}, L., {Robertson}, B., {Cox},
  T.~J., {van der Marel}, R.~P., \& {Alcock}, C. 2007, \apj, 668, 949

\bibitem[{{Forero-Romero} {et~al.}(2011){Forero-Romero}, {Hoffman}, {Yepes},
  {Gottl{\"o}ber}, {Piontek}, {Klypin}, \& {Steinmetz}}]{ForeroRomero2011}
{Forero-Romero}, J.~E., {Hoffman}, Y., {Yepes}, G., {Gottl{\"o}ber}, S.,
  {Piontek}, R., {Klypin}, A., \& {Steinmetz}, M. 2011, \mnras, 417, 1434

\bibitem[{{Fouquet} {et~al.}(2012){Fouquet}, {Hammer}, {Yang}, {Puech}, \&
  {Flores}}]{Fouquet2012}
{Fouquet}, S., {Hammer}, F., {Yang}, Y., {Puech}, M., \& {Flores}, H. 2012,
  \mnras, 427, 1769

\bibitem[{{Freeman} \& {Bland-Hawthorn}(2002)}]{Freeman2002}
{Freeman}, K., \& {Bland-Hawthorn}, J. 2002, \araa, 40, 487

\bibitem[{{Gottloeber} {et~al.}(2010){Gottloeber}, {Hoffman}, \&
  {Yepes}}]{clues2010}
{Gottloeber}, S., {Hoffman}, Y., \& {Yepes}, G. 2010, ArXiv e-prints

\bibitem[{{Hoffman} {et~al.}(2012){Hoffman}, {Metuki}, {Yepes},
  {Gottl{\"o}ber}, {Forero-Romero}, {Libeskind}, \& {Knebe}}]{Hoffman12}
{Hoffman}, Y., {Metuki}, O., {Yepes}, G., {Gottl{\"o}ber}, S., {Forero-Romero},
  J.~E., {Libeskind}, N.~I., \& {Knebe}, A. 2012, \mnras, 425, 2049

\bibitem[{{Jarosik} {et~al.}(2011){Jarosik}, {Bennett}, {Dunkley}, {Gold},
  {Greason}, {Halpern}, {Hill}, {Hinshaw}, {Kogut}, {Komatsu}, {Larson},
  {Limon}, {Meyer}, {Nolta}, {Odegard}, {Page}, {Smith}, {Spergel}, {Tucker},
  {Weiland}, {Wollack}, \& {Wright}}]{Jarosik2011}
{Jarosik}, N., {Bennett}, C.~L., {Dunkley}, J., {Gold}, B., {Greason}, M.~R.,
  {Halpern}, M., {Hill}, R.~S., {Hinshaw}, G., {Kogut}, A., {Komatsu}, E.,
  {Larson}, D., {Limon}, M., {Meyer}, S.~S., {Nolta}, M.~R., {Odegard}, N.,
  {Page}, L., {Smith}, K.~M., {Spergel}, D.~N., {Tucker}, G.~S., {Weiland},
  J.~L., {Wollack}, E., \& {Wright}, E.~L. 2011, \apjs, 192, 14

\bibitem[{{Karachentsev} {et~al.}(2004){Karachentsev}, {Karachentseva},
  {Huchtmeier}, \& {Makarov}}]{Karachentsev04}
{Karachentsev}, I.~D., {Karachentseva}, V.~E., {Huchtmeier}, W.~K., \&
  {Makarov}, D.~I. 2004, \aj, 127, 2031

\bibitem[{{Kazantzidis} {et~al.}(2008){Kazantzidis}, {Bullock}, {Zentner},
  {Kravtsov}, \& {Moustakas}}]{Kazantzidis2008}
{Kazantzidis}, S., {Bullock}, J.~S., {Zentner}, A.~R., {Kravtsov}, A.~V., \&
  {Moustakas}, L.~A. 2008, \apj, 688, 254

\bibitem[{{Klypin} {et~al.}(1999){Klypin}, {Kravtsov}, {Valenzuela}, \&
  {Prada}}]{Klypin99}
{Klypin}, A., {Kravtsov}, A.~V., {Valenzuela}, O., \& {Prada}, F. 1999, \apj,
  522, 82

\bibitem[{{Klypin} {et~al.}(2011){Klypin}, {Trujillo-Gomez}, \&
  {Primack}}]{Bolshoi}
{Klypin}, A.~A., {Trujillo-Gomez}, S., \& {Primack}, J. 2011, \apj, 740, 102

\bibitem[{{Knebe} {et~al.}(2011){Knebe}, {Libeskind}, {Doumler}, {Yepes},
  {Gottl{\"o}ber}, \& {Hoffman}}]{Knebe2011}
{Knebe}, A., {Libeskind}, N.~I., {Doumler}, T., {Yepes}, G., {Gottl{\"o}ber},
  S., \& {Hoffman}, Y. 2011, \mnras, 417, L56

\bibitem[{{Libeskind} {et~al.}(2012){Libeskind}, {Hoffman}, {Forero-Romero},
  {Gottl{\"o}ber}, {Knebe}, {Steinmetz}, \& {Klypin}}]{Libeskind12}
{Libeskind}, N.~I., {Hoffman}, Y., {Forero-Romero}, J., {Gottl{\"o}ber}, S.,
  {Knebe}, A., {Steinmetz}, M., \& {Klypin}, A. 2012, ArXiv e-prints

\bibitem[{{Liu} {et~al.}(2011){Liu}, {Gerke}, {Wechsler}, {Behroozi}, \&
  {Busha}}]{Liu2011}
{Liu}, L., {Gerke}, B.~F., {Wechsler}, R.~H., {Behroozi}, P.~S., \& {Busha},
  M.~T. 2011, \apj, 733, 62

\bibitem[{{McConnachie} {et~al.}(2009){McConnachie}, {Irwin}, {Ibata},
  {Dubinski}, {Widrow}, {Martin}, {C{\^o}t{\'e}}, {Dotter}, {Navarro},
  {Ferguson}, {Puzia}, {Lewis}, {Babul}, {Barmby}, {Bienaym{\'e}}, {Chapman},
  {Cockcroft}, {Collins}, {Fardal}, {Harris}, {Huxor}, {Mackey},
  {Pe{\~n}arrubia}, {Rich}, {Richer}, {Siebert}, {Tanvir}, {Valls-Gabaud}, \&
  {Venn}}]{pandas09}
{McConnachie}, A.~W., {Irwin}, M.~J., {Ibata}, R.~A., {Dubinski}, J., {Widrow},
  L.~M., {Martin}, N.~F., {C{\^o}t{\'e}}, P., {Dotter}, A.~L., {Navarro},
  J.~F., {Ferguson}, A.~M.~N., {Puzia}, T.~H., {Lewis}, G.~F., {Babul}, A.,
  {Barmby}, P., {Bienaym{\'e}}, O., {Chapman}, S.~C., {Cockcroft}, R.,
  {Collins}, M.~L.~M., {Fardal}, M.~A., {Harris}, W.~E., {Huxor}, A., {Mackey},
  A.~D., {Pe{\~n}arrubia}, J., {Rich}, R.~M., {Richer}, H.~B., {Siebert}, A.,
  {Tanvir}, N., {Valls-Gabaud}, D., \& {Venn}, K.~A. 2009, \nat, 461, 66

\bibitem[{{Moore} {et~al.}(1999){Moore}, {Ghigna}, {Governato}, {Lake},
  {Quinn}, {Stadel}, \& {Tozzi}}]{Moore99}
{Moore}, B., {Ghigna}, S., {Governato}, F., {Lake}, G., {Quinn}, T., {Stadel},
  J., \& {Tozzi}, P. 1999, \apjl, 524, L19

\bibitem[{{Peebles} \& {Nusser}(2010)}]{Peebles2010}
{Peebles}, P.~J.~E., \& {Nusser}, A. 2010, \nat, 465, 565

\bibitem[{{Purcell} \& {Zentner}(2012)}]{Purcell2012}
{Purcell}, C.~W., \& {Zentner}, A.~R. 2012, {JCAP}, 12, 7

\bibitem[{{Ribas} {et~al.}(2005){Ribas}, {Jordi}, {Vilardell}, {Fitzpatrick},
  {Hilditch}, \& {Guinan}}]{ribas05}
{Ribas}, I., {Jordi}, C., {Vilardell}, F., {Fitzpatrick}, E.~L., {Hilditch},
  R.~W., \& {Guinan}, E.~F. 2005, \apjl, 635, L37

\bibitem[{{Riebe} {et~al.}(2011){Riebe}, {Partl}, {Enke}, {Forero-Romero},
  {Gottloeber}, {Klypin}, {Lemson}, {Prada}, {Primack}, {Steinmetz}, \&
  {Turchaninov}}]{2011arXiv1109.0003R}
{Riebe}, K., {Partl}, A.~M., {Enke}, H., {Forero-Romero}, J., {Gottloeber}, S.,
  {Klypin}, A., {Lemson}, G., {Prada}, F., {Primack}, J.~R., {Steinmetz}, M.,
  \& {Turchaninov}, V. 2011, ArXiv e-prints

\bibitem[{{Teyssier} {et~al.}(2012){Teyssier}, {Johnston}, \&
  {Kuhlen}}]{Teyssier2012}
{Teyssier}, M., {Johnston}, K.~V., \& {Kuhlen}, M. 2012, \mnras, 426, 1808

\bibitem[{{Tikhonov} \& {Klypin}(2009)}]{Anton09}
{Tikhonov}, A.~V., \& {Klypin}, A. 2009, \mnras, 395, 1915

\bibitem[{{Tollerud} {et~al.}(2011){Tollerud}, {Boylan-Kolchin}, {Barton},
  {Bullock}, \& {Trinh}}]{Tollerud2011}
{Tollerud}, E.~J., {Boylan-Kolchin}, M., {Barton}, E.~J., {Bullock}, J.~S., \&
  {Trinh}, C.~Q. 2011, \apj, 738, 102

\bibitem[{{van der Marel} {et~al.}(2012){van der Marel}, {Fardal}, {Besla},
  {Beaton}, {Sohn}, {Anderson}, {Brown}, \& {Guhathakurta}}]{vanderMarel12}
{van der Marel}, R.~P., {Fardal}, M., {Besla}, G., {Beaton}, R.~L., {Sohn},
  S.~T., {Anderson}, J., {Brown}, T., \& {Guhathakurta}, P. 2012, \apj, 753, 8

\bibitem[{{van der Marel} \& {Guhathakurta}(2008)}]{vanderMarel08}
{van der Marel}, R.~P., \& {Guhathakurta}, P. 2008, \apj, 678, 187

\end{thebibliography}


\end{document}

